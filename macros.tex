%% Additional packages and commands to be placed in the preamble

%%%%%%%%%%%%%%%%%%%%%%%%%%%%%%%%%%%%%%%%%%%%%%%%%%%%%%%%%%%%%%%%%%%%%%%%%%%%%%%%%%%%%%%%
%% Packages
\usepackage{amsmath}
\usepackage{amssymb}
\usepackage{framed} %% debugging: boxes around figures to determine their size
%\usepackage{caption} % probably already loaded in emnlp2021.sty
% PW: is it allowed to descrease size between image and caption? \captionsetup[figure]{skip=5pt} % default is 10pt?
\usepackage{subcaption}  %% needed to have subfigures
\usepackage{tabularx}  %% X column for tables: automatic line breaks
\usepackage{booktabs}  %% nice tables: \toprule, \bottomrule, \midrule
\usepackage{multirow}  %% multirow in table with \multirow{number rows}{width}{text}
%\usepackage{tablefootnote}  %% for footnote to work in tables \tablefootnote[...]{...}

% make siunitx and gb4e work together: https://tex.stackexchange.com/questions/364163/incompatibility-between-siunitx-and-gb4e
\makeatletter
\def\new@fontshape{}
\makeatother
\usepackage{siunitx} %% formatting numbers/units. used in appendix for detailed eval results formatting

\usepackage{xspace}  % \xspace command. add space after newcommand or don't (context-dependent) 
%% define font to be one point smaller (used for lform in typewriter) see https://tex.stackexchange.com/a/365025
%\makeatletter
%\newcommand{\oneptsmaller}[1]{%
%  \begingroup
%  \fontsize{\dimexpr\f@size pt-1pt}{\f@baselineskip}\selectfont
%  #1%
%  \endgroup
%}
%\makeatother

\usepackage{tikz}  %% drawing
\usetikzlibrary{shapes.geometric, arrows}
\usetikzlibrary{calc, positioning}  % for bars in table
\usepackage{forest}  %% drawing linguistic trees: used for AM algebra figure: the AM term and the AM dep-tree and also for the PPrecursion/subjPP trees
\useforestlibrary{linguistics}
\forestapplylibrarydefaults{linguistics}

\usepackage{stmaryrd} %% denotation brackets (Lucia)

\usepackage[compress,sort,capitalise]{cleveref}  %% automatically insert type of reference: Figure/Table/Section...
  %% NB: should be placed after hyperref is loaded
\usepackage{hyperref}
\usepackage{float}

\usepackage[inline]{enumitem}  %% customizing enumerations/lists: label=(\arabic*) or \Alph* ... or \roman*  , inliste list with starred version of normal lists: \begin{enumerate*}

\usepackage{gb4e}  % for linguistic examples \begin{exe} \ex example1 \ex example 2 \end{exe}
  %% ATTENTION: allows _ and ^ in ordinary text: can conflict with other commands/packages! 
  %% therefore use \noautomath directly after package loaded. see also https://tex.stackexchange.com/q/101300
\noautomath

%%%%%%%%%%%%%%%%%%%%%%%%%%%%%%%%%%%%%%%%%%%%%%%%%%%%%%%%%%%%%%%%%%%%%%%%%%%%%%%%%%%%%%%%%%
%% Defining own commands
\newcommand{\todocite}{\todo{citation!}}
\newcommand{\sortof}[1]{`#1'}
%\newcommand{\lform}[1]{{\fontfamily{cmtt}\selectfont #1}}
\newcommand{\lform}[1]{\texttt{#1}}
%\newcommand{\lform}[1]{\oneptsmaller{\texttt{#1}}}
%\newcommand{\graphedge}[1]{\textsl{#1}}
\newcommand{\graphedge}[1]{`#1'}
\newcommand{\graphnode}[1]{`#1'}
\newcommand{\word}[1]{``#1''}
\newcommand{\phrase}[1]{``#1''}
\newcommand{\sentence}[1]{``#1''}

%\crefname{section}{sec.}{secs.} % see page 14 (sec 8.1.2) in the cleveref documentation
%\crefname{section}{tab.}{tabs.}
% make cleverref aware of examples and cite them with sourrounding parenthesis: see https://tex.stackexchange.com/a/318983
% if we can't use cleveref we could define a command \newcommand{\pref}[1]{(\ref{#1})} like suggested in https://tex.stackexchange.com/a/211871
\crefname{xnumi}{}{}
\creflabelformat{xnumi}{(#2#1#3)}
\crefrangeformat{xnumi}{(#3#1#4)--(#5#2#6)}

% tikz node and edge style for graphs
%  normal node
\tikzstyle{nnode} = [ellipse, text centered, draw=black, inner sep=2pt] %, outer sep=0pt
%  determiner node
\tikzstyle{dnode} = [rectangle, rounded corners, text centered, draw=black] % minimum width=3cm, minimum height=1cm, , fill=red!30
\tikzstyle{arrow} = [thick,->,>=stealth]

% for reference: the color values used in lindeman-etal-2019-compositional
%\colorlet{colorwant}{green!30}
%\colorlet{colorboy}{brown!10!yellow!60}
%\colorlet{colorthe}{blue!30}
%\colorlet{colorgo}{red!30}
% new colors should help colorblind, but actually more an excuse, just wanted to experiment with colors ;) - pw
\definecolor{colorwant}{HTML}{ABBEF1} % lighter blue tone
\definecolor{colorgo}{HTML}{DE8BB3} % red-pink tone
\definecolor{colorthe}{HTML}{F1AD83} % light orange tone
\definecolor{colorboy}{HTML}{FFE19D} % light yellow tone

% AM algebra things: %%%%%%%%%%%
\newcommand{\sources}{\ensuremath{\mathcal{S}}}
\newcommand{\tp}[1]{\ensuremath{\tau(#1)}}   % type variable
\newcommand{\type}[1]{\ensuremath{[#1]}}

\newcommand{\app}[1]{\text{\textsc{App}}\ensuremath{_{#1}}\xspace}
\newcommand{\modify}[1]{\text{\textsc{mod}}\ensuremath{_{#1}}\xspace}
\newcommand{\lignore}{\text{\textsc{ignore}}}
\newcommand{\of}[1]{\left(#1\right)}

% source
\newcommand{\src}[1]{\text{\textsc{#1}}\xspace} % source name 
%\newcommand{\obj}[1][]{\text{\src{o}}\ensuremath{_{#1}}} % object sources. \obj or \obj[i]
\newcommand{\subj}{\text{\src{s}}} % subject source, just \subj
% \newcommand{\op}[1][i]{\text{\src{op}}\ensuremath{_{#1}}} % op sources \op[i]
\newcommand{\rt}{\src{rt}} % root source
% with color red:
\newcommand{\srcr}[1]{\textcolor{red}{\src{#1}}} % source name 

\newcommand{\G}[1]{\ensuremath{G_{\text{#1}}}} % naming graphs


%% Colored bars as background in tables using \asbar
% this is based on TeX code found on arXiv for the Ontanon et al paper: https://arxiv.org/abs/2108.04378
\pgfdeclarelayer{background}
\pgfdeclarelayer{main}
\pgfsetlayers{background,main}
\newcommand*{\minWidth}{25}  % controls size of total box
\newcommand*{\maxValue}{100}
\newcommand{\makeBar}[3]{%
  \tikz[baseline]{
    \node[anchor=base,text width=\minWidth,align=#2,inner sep=0pt,inner xsep=4pt,outer sep=0pt] (n) {\strut{\hfill#1}};  % ,draw=black % \hfill makes numbers right-aligned
    \begin{pgfonlayer}{background}
        {
       \edef\color{#3}
       \pgfmathparse{abs(#1/\maxValue)}
       \edef\contents{{\pgfmathresult}}
       \fill[font=\boldmath,color=\color] (n.north west) rectangle ($(n.south west)!{{\contents}}!(n.south east)$);  %% that's the box for the actual percentage
       \fill[font=\boldmath,color=cyan!10] ($(n.north west)!{{\contents}}!(n.north east)$) rectangle (n.south east);  % that's the box for the rest
       }
    \end{pgfonlayer}
  }
}
\newcommand{\asbar}[1]{\makeBar{#1}{center}{cyan!25}}
\newcommand{\stderr}[1]{\tiny$\pm$#1}
% \newcommand{\stderr}[1]{\footnotesize{$\pm$#1}}
\newcommand{\pmgray}{\textcolor{gray}{$\pm$}}  % sign in between columns separating mean and standard dev

% Class names 'Lexicalized'/'ObjProper'/'Structural'
\newcommand{\genclass}[1]{\textsc{#1}\xspace} % 
\newcommand{\lexg}{\genclass{Lex}}
\newcommand{\propg}{\genclass{Prop}}
\newcommand{\structg}{\genclass{Struct}}

% how should we call AM parser models with and without BERT?
% NOTE: don't forget to change it in the recursion depth plot too!!!
\newcommand{\amBert}{AM+B} %AM+BERT  
\newcommand{\amToken}{AM} %AM-BERT aka AM Token
\newcommand{\dist}{dist\xspace} %with distance info added

\newcommand{\oldNumber}[1]{\textcolor{black!40}{#1}}

% Author comments
\usepackage{color}
\newcommand\bmmax{0} % magic to avoid 'too many math alphabets' error
\usepackage{bm}
\definecolor{orange}{rgb}{1,0.5,0}
\definecolor{mdgreen}{rgb}{0.05,0.6,0.05}
\definecolor{mdblue}{rgb}{0,0,0.7}
\definecolor{dkblue}{rgb}{0,0,0.5}
\definecolor{dkgray}{rgb}{0.3,0.3,0.3}
\definecolor{slate}{rgb}{0.25,0.25,0.4}
\definecolor{gray}{rgb}{0.5,0.5,0.5}
\definecolor{ltgray}{rgb}{0.7,0.7,0.7}
\definecolor{purple}{rgb}{0.7,0,1.0}
\definecolor{lavender}{rgb}{0.65,0.55,1.0}
\definecolor{brown}{rgb}{0.6,0.2,0.2}

\newcommand{\ensuretext}[1]{#1}
\newcommand{\ldmarker}{\ensuretext{\textcolor{purple}{\ensuremath{^{\textsc{L}}_{\textsc{D}}}}}}
\newcommand{\pwmarker}{\ensuretext{\textcolor{magenta}{\ensuremath{^{\textsc{P}}_{\textsc{W}}}}}}
\newcommand{\yymarker}{\ensuretext{\textcolor{mdgreen}{\ensuremath{^{\textsc{Y}}_{\textsc{Y}}}}}}
\newcommand{\jgmarker}{\ensuretext{\textcolor{orange}{\ensuremath{^{\textsc{J}}_{\textsc{G}}}}}}
\newcommand{\akmarker}{\ensuretext{\textcolor{blue}{\ensuremath{^{\textsc{A}}_{\textsc{K}}}}}}
\newcommand{\codemarker}{\ensuretext{\textcolor{brown}{CODE}}}


\newcommand{\arkcomment}[3]{\ensuretext{\textcolor{#3}{[#1 #2]}}}
%\renewcommand{\arkcomment}[3]{}
%\renewcommand{\todo}[1]{}
\newcommand{\ld}[1]{\arkcomment{\ldmarker}{#1}{purple}}
\newcommand{\pw}[1]{\arkcomment{\pwmarker}{#1}{magenta}}
\newcommand{\yy}[1]{\arkcomment{\yymarker}{#1}{mdgreen}}
\newcommand{\jg}[1]{\arkcomment{\jgmarker}{#1}{orange}}
\newcommand{\ak}[1]{\arkcomment{\akmarker}{#1}{blue}}
\newcommand{\code}[1]{}%\arkcomment{\codemarker}{#1}{brown}}% marks discrepancies between text and code. Change this to {} for print version


\newcommand{\ignore}[1]{}
